\hypertarget{README_8md_source}{}\section{R\+E\+A\+D\+M\+E.\+md}

\begin{DoxyCode}
00001 # 2018-wannabe-postdoc-1
00002 
00003 This is the repository for problem 1 of the 2018 BDAthlon programming contest.
00004 
00005 It my solution for the problem on genetic technology mapping.
00006 
00007 
00008 ## Organization of the Repository
00009 
00010 + docs
00011    - Documentation generated by documentation generators
00012        * Doxygen
00013        * pydoc (for Python scripts): /usr/bin/pydoc
00014 + notes
00015    - Software licenses
00016        * *MIT License*.
00017    - Guidelines for collaborating on open source software and/or hardware
00018       projects.
00019     * Documentation about guidelines that I am following for my research,
00020          and for my research collaborators to know about.
00021   - Externalities list.
00022     * *Publicly available library, API, or framework* that I have used as
00023         external components for my software.
00024        * Some *Python* modules from the repository
       [bibtex-analytics](https://github.com/eda-ricercatore/bibtex-analytics),
00025            which I developed.
00026 
00027 
00028 ##  Description of the Software Solution for Genetic Technology Mapping
00029 
00030 My solution for genetic technology mapping, **problem1\_solution.py**, takes in
00031    two input parameters, **[input JSON netlist]** and
00032    **[output JSON technology mapping]**.
00033 
00034 The first input parameter **[input JSON netlist]** is a JSON file that contains
00035    a structural netlist for genetic technology mapping.
00036 
00037 The other input parameter **[output JSON technology mapping]** is the filename
00038    of an output JSON file that contains a genetic technology mapping for the
00039    input structural netlist (**[input JSON netlist]**).
00040 
00041 The front-end of **problem1\_solution.py** parses the input arguments, checks
00042    their validity, and parses the input structural netlist for genetic
00043    technology mapping.
00044    It also parses the genetic technology library **genetic\_gate\_library.json**,
00045        maps each genetic NOT gate into a *genetic\_not\_gate* object, and stores
00046        each object in a map of *(id,objects)*;
00047        the *id* of the object is the *id* field of the genetic NOT gate, which is
00048            specified in the genetic technology library.
00049 
00050 The genetic technology mapping engine of **problem1\_solution.py** consists of a
00051    (set of) solution(s) to perform genetic technology mapping.
00052    A solution is an implementation of a known/new algorithm/heuristic for
00053        genetic technology mapping.
00054 
00055 **Dr. Nicholas Roehner and Dr. Curtis Madsen, I am running out of time, so I
00056    will sketch the outline of my solutions, and update them as I solve the problem.**
00057 
00058 Solution *1a* performs *brute force search* to explore different options for
00059    genetic technology mapping.
00060 +  Explore each permutation of NOT gates.
00061 +  For each selection, store its alpha value and the permutation in a table
00062 + Enumerate a table to find the largest alpha value, and select the
00063        corresponding permutation of NOT gates.
00064 
00065 
00066 Solution *1b* uses simulated annealing for discrete optimization.
00067 + Pseudo-randomly select a permutation of NOT gates.
00068 + Initialize temperature to be very hot
00069 + While temperature is not 0, cool the temperature of the annealing process.
00070    - As the temperature cools, slowly decrease the
00071 
00072 Solution *1b* uses a genetic algorithm for discrete optimization.
00073 
00074 
00075 Other solutions considered:
00076 + 0-1 integer linear programming (ILP)
00077    * Can't formulate the objective function of the 0-1 ILP problem.
00078 + pseudo-boolan optimization (PBO)
00079    * Can't formulate the conjunctive normal form (CNF) boolean satisfiability
00080        formula for PBO.
00081 
00082 The back-end of **problem1\_solution.py** generates an output file containing
00083    the genetic technology mapping of the input genetic circuit, in JSON format.
00084 
00085 
00086 
00087 
00088 
00089 
00090 
00091 
00092 ##  Instructions on How to Build and Run the Software Solution
00093 
00094 ###    Building and Executing the Software Solution
00095 
00096 To execute the software solution, try:
00097 
00098    ./problem1\_solution.py [input JSON netlist] [output JSON technology mapping]
00099 
00100 Input Parameters:
00101 [input JSON netlist]:                          A JSON file that contains a structural
00102                                                                        netlist for genetic technology
       mapping.
00103 
00104 [output JSON technology mapping]:  A filename of an output JSON file that
00105                                                                        contains a genetic technology
       mapping for
00106                                                                        the input structural netlist.
00107 
00108 
00109 ###    Documentation Generation
00110 
00111 To use *Doxygen* to generate documentation for the Python software, try:
00112 
00113    make doxygen
00114 
00115 To view the *Doxygen*-generated documention, open the file
       [**docs/html/index.html**](https://github.com/BDAthlon/2018-wannabe-postdoc-1/blob/master/docs/html/index.html) in a Web browser.
00116 
00117 The command **make doxygeninit** has already been used to generate a *Doxygen*
00118    configuration file named **doxygen.config**.
00119    **DO NOT RUN THE COMMAND *doxygen.config* AGAIN!!!**
00120 
00121 To use *pydoc* to view or generate documentation for my software solution
00122    (Python code), try:
00123 
00124    pydoc [*Python* package/module/class]
00125 
00126 A *Python* package corresponds to a subdirectory of this repository, while a
00127    *Python* module/class corresponds to a *Python* source file in a subdirectory.
00128 
00129 
00130 
00131 
00132 ## Miscellaneous
00133 
00134 ###    Refactoring attempt: *Utilities* package
00135 
00136 Refactor the class **queue\_ip\_args** in the *Python* module
00137    *queue\_ip\_arguments.py*, so that does not need the argument **which\_script**
00138    for the static method **set\_input\_arguments(list\_of\_ip\_arguments,which\_script)**.
00139    That is, refactor the static method to **set\_input\_arguments(list\_of\_ip\_arguments)**.
00140 
00141 Using this refactored *Python* class has an impact on its corresponding
00142    *Python* class in the **[bibtex-analytics](https://github.com/eda-ricercatore/bibtex-analytics)**
00143    software.
00144    Use this refactored code to handle usage modes in the **bibtex-analytics**
00145        software, such as: obtaining a sorted list of keyphrases/keywords; obtaining
00146        a sorted list of publishers, organizations, and institutions; and a list of
00147        series of books or conference proceedings.
00148    These modes shall be specified by an input argument.
00149    If multiple modes are specified as input arguments, an order of precedance
00150        is used to specify which mode shall be selected for processing.
00151    The **[bibtex-analytics](https://github.com/eda-ricercatore/bibtex-analytics)**
00152        software can only process one mode per execution run.
00153 
00154 
00155 #  References
00156 
00157 Citations/References that use the LaTeX/BibTeX notation are taken from my
00158    BibTeX database (set of BibTeX entries).
00159 
00160 Additional references not found in the reference list shall be indicated below
00161    (TO BE UPDATED).
00162 
00163 
00164    @misc\{Roehner2018,
00165        Address = \{Boston, \{MA\}\},
00166        Author = \{Nicholas Roehner and Curtis Madsen\},
00167        Howpublished = \{Self-published\},
00168        Month = \{July 31\},
00169        Publisher = \{Boston University\},
00170        School = \{Boston University\},
00171        Title = \{BDAthlon 2018\},
00172        Year = \{2018\}\}
00173 
00174 
00175 
00176 #  Author Information
00177 
00178 The MIT License (MIT)
00179 
00180 Copyright (c) <2018> Zhiyang Ong
00181 
00182 Permission is hereby granted, free of charge, to any person obtaining a copy of this software and
       associated documentation files (the "Software"), to deal in the Software without restriction, including without
       limitation the rights to use, copy, modify, merge, publish, distribute, sublicense, and/or sell copies of
       the Software, and to permit persons to whom the Software is furnished to do so, subject to the following
       conditions:
00183 
00184 The above copyright notice and this permission notice shall be included in all copies or substantial
       portions of the Software.
00185 
00186 THE SOFTWARE IS PROVIDED "AS IS", WITHOUT WARRANTY OF ANY KIND, EXPRESS OR IMPLIED, INCLUDING BUT NOT
       LIMITED TO THE WARRANTIES OF MERCHANTABILITY, FITNESS FOR A PARTICULAR PURPOSE AND NONINFRINGEMENT. IN NO
       EVENT SHALL THE AUTHORS OR COPYRIGHT HOLDERS BE LIABLE FOR ANY CLAIM, DAMAGES OR OTHER LIABILITY, WHETHER IN
       AN ACTION OF CONTRACT, TORT OR OTHERWISE, ARISING FROM, OUT OF OR IN CONNECTION WITH THE SOFTWARE OR THE USE
       OR OTHER DEALINGS IN THE SOFTWARE.
00187 
00188 Email address: echo "cukj -wb- 23wU4X5M589 TROJANS cqkH wiuz2y 0f Mw Stanford" | awk '\{
       sub("23wU4X5M589","F.d\_c\_b. ") sub("Stanford","d0mA1n"); print $5, $2, $8; for (i=1; i<=1; i++) print "6\(\backslash\)b"; print $9, $7,
       $6 \}' | sed y/kqcbuHwM62z/gnotrzadqmC/ | tr 'q' ' ' | tr -d [:cntrl:] | tr -d 'ir' | tr y "\(\backslash\)n"      Don't
       compromise my computing accounts. You have been warned.
\end{DoxyCode}
