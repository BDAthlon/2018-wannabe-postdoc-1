This is the repository for problem 1 of the 2018 B\+D\+Athlon programming contest.

It my solution for the problem on genetic technology mapping.

\subsection*{Organization of the Repository}


\begin{DoxyItemize}
\item docs
\begin{DoxyItemize}
\item Documentation generated by documentation generators
\begin{DoxyItemize}
\item Doxygen
\item pydoc (for Python scripts)\+: /usr/bin/pydoc
\end{DoxyItemize}
\end{DoxyItemize}
\item notes
\begin{DoxyItemize}
\item Software licenses
\begin{DoxyItemize}
\item {\itshape M\+I\+T License}.
\end{DoxyItemize}
\item Guidelines for collaborating on open source software and/or hardware projects.
\item Documentation about guidelines that I am following for my research, and for my research collaborators to know about.
\end{DoxyItemize}
\end{DoxyItemize}

Externalities list.
\begin{DoxyItemize}
\item {\itshape Publicly available library, A\+P\+I, or framework} that I have used as external components for my software.
\begin{DoxyItemize}
\item Some {\itshape Python} modules from the repository \href{https://github.com/eda-ricercatore/bibtex-analytics}{\tt bibtex-\/analytics}, which I developed.
\end{DoxyItemize}
\end{DoxyItemize}

\subsection*{Description of the Software Solution for Genetic Technology Mapping}

My solution for genetic technology mapping, {\bfseries problem1\+\_\+solution.\+py}, takes in two input parameters, $\ast$$\ast$\mbox{[}input J\+S\+O\+N netlist\mbox{]}$\ast$$\ast$ and $\ast$$\ast$\mbox{[}output J\+S\+O\+N technology mapping\mbox{]}$\ast$$\ast$.

The first input parameter $\ast$$\ast$\mbox{[}input J\+S\+O\+N netlist\mbox{]}$\ast$$\ast$ is a J\+S\+O\+N file that contains a structural netlist for genetic technology mapping.

The other input parameter $\ast$$\ast$\mbox{[}output J\+S\+O\+N technology mapping\mbox{]}$\ast$$\ast$ is the filename of an output J\+S\+O\+N file that contains a genetic technology mapping for the input structural netlist ($\ast$$\ast$\mbox{[}input J\+S\+O\+N netlist\mbox{]}$\ast$$\ast$).

The front-\/end of {\bfseries problem1\+\_\+solution.\+py} parses the input arguments, checks their validity, and parses the input structural netlist for genetic technology mapping. It also parses the genetic technology library {\bfseries genetic\+\_\+gate\+\_\+library.\+json}, maps each genetic N\+O\+T gate into a {\itshape genetic\+\_\+not\+\_\+gate} object, and stores each object in a map of $\ast$(id,objects)$\ast$; the {\itshape id} of the object is the {\itshape id} field of the genetic N\+O\+T gate, which is specified in the genetic technology library.

The genetic technology mapping engine of {\bfseries problem1\+\_\+solution.\+py} consists of a (set of) solution(s) to perform genetic technology mapping. A solution is an implementation of a known/new algorithm/heuristic for genetic technology mapping.

{\bfseries Dr. Nicholas Roehner and Dr. Curtis Madsen, I am running out of time, so I will sketch the outline of my solutions, and update them as I solve the problem.}

Solution {\itshape 1a} performs {\itshape brute force search} to explore different options for genetic technology mapping.
\begin{DoxyItemize}
\item Explore each permutation of N\+O\+T gates.
\item For each selection, store its alpha value and the permutation in a table
\item Enumerate a table to find the largest alpha value, and select the corresponding permutation of N\+O\+T gates.
\end{DoxyItemize}

Solution {\itshape 1b} uses simulated annealing for discrete optimization.
\begin{DoxyItemize}
\item Pseudo-\/randomly select a permutation of N\+O\+T gates.
\item Initialize temperature to be very hot
\item While temperature is not 0, cool the temperature of the annealing process.
\begin{DoxyItemize}
\item As the temperature cools, slowly decrease the
\end{DoxyItemize}
\end{DoxyItemize}

Solution {\itshape 1b} uses a genetic algorithm for discrete optimization.

Other solutions considered\+:
\begin{DoxyItemize}
\item 0-\/1 integer linear programming (I\+L\+P)
\begin{DoxyItemize}
\item Can\textquotesingle{}t formulate the objective function of the 0-\/1 I\+L\+P problem.
\end{DoxyItemize}
\item pseudo-\/boolan optimization (P\+B\+O)
\begin{DoxyItemize}
\item Can\textquotesingle{}t formulate the conjunctive normal form (C\+N\+F) boolean satisfiability formula for P\+B\+O.
\end{DoxyItemize}
\end{DoxyItemize}

The back-\/end of {\bfseries problem1\+\_\+solution.\+py} generates an output file containing the genetic technology mapping of the input genetic circuit, in J\+S\+O\+N format.

\subsection*{Instructions on How to Build and Run the Software Solution}

\subsubsection*{Building and Executing the Software Solution}

To execute the software solution, try\+: \begin{DoxyVerb}./problem1_solution.py [input JSON netlist] [output JSON technology mapping]
\end{DoxyVerb}


Input Parameters\+: \mbox{[}input J\+S\+O\+N netlist\mbox{]}\+: A J\+S\+O\+N file that contains a structural netlist for genetic technology mapping.

\mbox{[}output J\+S\+O\+N technology mapping\mbox{]}\+: A filename of an output J\+S\+O\+N file that contains a genetic technology mapping for the input structural netlist.

\subsubsection*{Documentation Generation}

To use {\itshape Doxygen} to generate documentation for the Python software, try\+: \begin{DoxyVerb}make doxygen
\end{DoxyVerb}


To view the {\itshape Doxygen}-\/generated documention, open the file \href{https://github.com/BDAthlon/2018-wannabe-postdoc-1/blob/master/docs/html/index.html}{\tt $\ast$$\ast$docs/html/index.html$\ast$$\ast$} in a Web browser.

The command {\bfseries make doxygeninit} has already been used to generate a {\itshape Doxygen} configuration file named {\bfseries doxygen.\+config}. {\bfseries D\+O N\+O\+T R\+U\+N T\+H\+E C\+O\+M\+M\+A\+N\+D {\itshape doxygen.\+config} A\+G\+A\+I\+N!!!}

To use {\itshape pydoc} to view or generate documentation for my software solution (Python code), try\+:

pydoc \mbox{[}{\itshape Python} package/module/class\mbox{]}

A {\itshape Python} package corresponds to a subdirectory of this repository, while a {\itshape Python} module/class corresponds to a {\itshape Python} source file in a subdirectory.

\subsection*{Miscellaneous}

\subsubsection*{Refactoring attempt\+: {\itshape Utilities} package}

Refactor the class {\bfseries queue\+\_\+ip\+\_\+args} in the {\itshape Python} module {\itshape \hyperlink{queue__ip__arguments_8py}{queue\+\_\+ip\+\_\+arguments.\+py}}, so that does not need the argument {\bfseries which\+\_\+script} for the static method {\bfseries set\+\_\+input\+\_\+arguments(list\+\_\+of\+\_\+ip\+\_\+arguments,which\+\_\+script)}. That is, refactor the static method to {\bfseries set\+\_\+input\+\_\+arguments(list\+\_\+of\+\_\+ip\+\_\+arguments)}.

Using this refactored {\itshape Python} class has an impact on its corresponding {\itshape Python} class in the $\ast$$\ast$\href{https://github.com/eda-ricercatore/bibtex-analytics}{\tt bibtex-\/analytics}$\ast$$\ast$ software. Use this refactored code to handle usage modes in the {\bfseries bibtex-\/analytics} software, such as\+: obtaining a sorted list of keyphrases/keywords; obtaining a sorted list of publishers, organizations, and institutions; and a list of series of books or conference proceedings. These modes shall be specified by an input argument. If multiple modes are specified as input arguments, an order of precedance is used to specify which mode shall be selected for processing. The $\ast$$\ast$\href{https://github.com/eda-ricercatore/bibtex-analytics}{\tt bibtex-\/analytics}$\ast$$\ast$ software can only process one mode per execution run.

\section*{References}

Citations/\+References that use the La\+Te\+X/\+Bib\+Te\+X notation are taken from my Bib\+Te\+X database (set of Bib\+Te\+X entries).

Additional references not found in the reference list shall be indicated below (T\+O B\+E U\+P\+D\+A\+T\+E\+D).

\begin{DoxyVerb}@misc{Roehner2018,
    Address = {Boston, {MA}},
    Author = {Nicholas Roehner and Curtis Madsen},
    Howpublished = {Self-published},
    Month = {July 31},
    Publisher = {Boston University},
    School = {Boston University},
    Title = {BDAthlon 2018},
    Year = {2018}}
\end{DoxyVerb}


\section*{Author Information}

The M\+I\+T License (M\+I\+T)

Copyright (c) $<$2018$>$ Zhiyang Ong

Permission is hereby granted, free of charge, to any person obtaining a copy of this software and associated documentation files (the \char`\"{}\+Software\char`\"{}), to deal in the Software without restriction, including without limitation the rights to use, copy, modify, merge, publish, distribute, sublicense, and/or sell copies of the Software, and to permit persons to whom the Software is furnished to do so, subject to the following conditions\+:

The above copyright notice and this permission notice shall be included in all copies or substantial portions of the Software.

T\+H\+E S\+O\+F\+T\+W\+A\+R\+E I\+S P\+R\+O\+V\+I\+D\+E\+D \char`\"{}\+A\+S I\+S\char`\"{}, W\+I\+T\+H\+O\+U\+T W\+A\+R\+R\+A\+N\+T\+Y O\+F A\+N\+Y K\+I\+N\+D, E\+X\+P\+R\+E\+S\+S O\+R I\+M\+P\+L\+I\+E\+D, I\+N\+C\+L\+U\+D\+I\+N\+G B\+U\+T N\+O\+T L\+I\+M\+I\+T\+E\+D T\+O T\+H\+E W\+A\+R\+R\+A\+N\+T\+I\+E\+S O\+F M\+E\+R\+C\+H\+A\+N\+T\+A\+B\+I\+L\+I\+T\+Y, F\+I\+T\+N\+E\+S\+S F\+O\+R A P\+A\+R\+T\+I\+C\+U\+L\+A\+R P\+U\+R\+P\+O\+S\+E A\+N\+D N\+O\+N\+I\+N\+F\+R\+I\+N\+G\+E\+M\+E\+N\+T. I\+N N\+O E\+V\+E\+N\+T S\+H\+A\+L\+L T\+H\+E A\+U\+T\+H\+O\+R\+S O\+R C\+O\+P\+Y\+R\+I\+G\+H\+T H\+O\+L\+D\+E\+R\+S B\+E L\+I\+A\+B\+L\+E F\+O\+R A\+N\+Y C\+L\+A\+I\+M, D\+A\+M\+A\+G\+E\+S O\+R O\+T\+H\+E\+R L\+I\+A\+B\+I\+L\+I\+T\+Y, W\+H\+E\+T\+H\+E\+R I\+N A\+N A\+C\+T\+I\+O\+N O\+F C\+O\+N\+T\+R\+A\+C\+T, T\+O\+R\+T O\+R O\+T\+H\+E\+R\+W\+I\+S\+E, A\+R\+I\+S\+I\+N\+G F\+R\+O\+M, O\+U\+T O\+F O\+R I\+N C\+O\+N\+N\+E\+C\+T\+I\+O\+N W\+I\+T\+H T\+H\+E S\+O\+F\+T\+W\+A\+R\+E O\+R T\+H\+E U\+S\+E O\+R O\+T\+H\+E\+R D\+E\+A\+L\+I\+N\+G\+S I\+N T\+H\+E S\+O\+F\+T\+W\+A\+R\+E.

Email address\+: echo \char`\"{}cukj -\/wb-\/ 23w\+U4\+X5\+M589 T\+R\+O\+J\+A\+N\+S cqk\+H wiuz2y 0f Mw Stanford\char`\"{} $\vert$ awk \textquotesingle{}\{ sub(\char`\"{}23w\+U4\+X5\+M589\char`\"{},\char`\"{}\+F.\+d\+\_\+c\+\_\+b. \char`\"{}) sub(\char`\"{}\+Stanford\char`\"{},\char`\"{}d0m\+A1n\char`\"{}); print \$5, \$2, \$8; for (i=1; i$<$=1; i++) print \char`\"{}6\textbackslash{}b\char`\"{}; print \$9, \$7, \$6 \}\textquotesingle{} $\vert$ sed y/kqcbu\+Hw\+M62z/gnotrzadqm\+C/ $\vert$ tr \textquotesingle{}q\textquotesingle{} \textquotesingle{} \textquotesingle{} $\vert$ tr -\/d \mbox{[}\+:cntrl\+:\mbox{]} $\vert$ tr -\/d \textquotesingle{}ir\textquotesingle{} $\vert$ tr y \char`\"{}\textbackslash{}n\char`\"{} Don\textquotesingle{}t compromise my computing accounts. You have been warned. 